% Presentation (Beamer) — "Модели поведения человека и их роль в экономическом развитии"
% Компиляция: xelatex presentation.tex (2 раза)
\documentclass[aspectratio=169]{beamer}
\usetheme{Madrid}
\usecolortheme{seagull}
\usefonttheme{professionalfonts}
\usepackage{fontspec}
\setmainfont{TeX Gyre Termes} % стандартный, хорошо выглядит в XeLaTeX
\usepackage{array}
\usepackage{ragged2e}
\usepackage{booktabs}
\usepackage{hyperref}
\hypersetup{colorlinks=true, urlcolor=blue}

\title[Модели поведения и развитие]{Модели поведения человека и их роль в экономическом развитии}
\author{KuzkaPim}
\institute{Лекция для студентов бакалавриата}
\date{\today}

\begin{document}
\begin{frame}
  \titlepage
\end{frame}

\begin{frame}{Введение: основные модели}
\centering
\begin{tabular}{p{0.35\linewidth} p{0.55\linewidth}}
\toprule
Модель & Краткое определение \\
\midrule
Рациональная & Акторы выбирают оптимум, максимизируя полезность. \\
Ограниченная рациональность & Ограниченные вычисления; эвристики и упрощения. \\
Поведенческая & Систематические отклонения от оптимума: предвзятости, эмоции. \\
Социальная/культурная & Решения зависят от норм, доверия и сетей. \\
\bottomrule
\end{tabular}
\note{Коротко объяснить каждую модель и подчеркнуть, что они могут сочетаться.}
\end{frame}

\begin{frame}{Роль моделей в экономике}
\centering
\begin{tabular}{p{0.22\linewidth} p{0.45\linewidth} p{0.25\linewidth}}
\toprule
Модель & Влияние на поведение экономики & Пример \\
\midrule
Рациональная & Прогнозируемые реакции на стимулы & Налоговые изменения \\
Ограниченная & Неполные реакции, фрикции & Внедрение технологий \\
Поведенческая & Пузырьки, иррациональный спрос & Кредитный бум \\
Социальная & Нормы формируют поведение & Сбережения, доверие \\
\bottomrule
\end{tabular}
\note{Подчеркнуть практическое значение понимания модели при проектировании политики.}
\end{frame}

\begin{frame}{Сравнительная таблица моделей}
\centering
\begin{tabular}{l c c c c}
\toprule
Критерий & Рациональная & Ограниченная & Поведенческая & Социальная \\
\midrule
Допущение & Максимизация & Эвристики & Систематич. ошибки & Нормы/сети \\
Прогнозируемость & Высокая & Средняя & Низкая/контекст & Низкая/культура \\
Рекомендации & Стимулы & Упрощение выбора & Нуджи / коррекция & Работа с нормами \\
\bottomrule
\end{tabular}
\note{Кратко пройтись по столбцам и привести примеры политик.}
\end{frame}

\begin{frame}{Модель рационального выбора}
\centering
\begin{tabular}{p{0.28\linewidth} p{0.58\linewidth}}
\toprule
Компонент & Описание / Последствия для развития \\
\midrule
Агент & Полностью рациональный; теоретическая эффективность \\
Информация & Полная/доступная; быстрое реагирование рынков \\
Политика & Налоги, субсидии как стимулы; ожидаемое изменение поведения \\
\bottomrule
\end{tabular}
\note{Отметить ограничения в реальности: неполная информация, транзакционные издержки.}
\end{frame}

\begin{frame}{Ограниченная рациональность и поведенческая экономика}
\centering
\begin{tabular}{p{0.28\linewidth} p{0.58\linewidth}}
\toprule
Явление & Механизм / Экономическое следствие \\
\midrule
Эвристики & Упрощённые правила → систематические ошибки \\
Предвзятость подтверждения & Отбор информации → медленная адаптация \\
Рамки выбора & Презентация влияет на решения (сбережения) \\
\bottomrule
\end{tabular}
\note{Привести примеры: пенсионные решения, кредитное поведение.}
\end{frame}

\begin{frame}{Социальные и культурные факторы}
\centering
\begin{tabular}{p{0.28\linewidth} p{0.58\linewidth}}
\toprule
Фактор & Последствие для развития \\
\midrule
Доверие & Снижает издержки транзакций → ускоряет инвестиции \\
Нормы & Поддерживают/препятствуют изменениям → важны для реформ \\
Сети & Распространение практик → скорость инноваций \\
\bottomrule
\end{tabular}
\note{Обсудить примеры: доверие vs коррупция; традиции сбережений.}
\end{frame}

\begin{frame}{Примеры и выводы}
\centering
\begin{tabular}{p{0.32\linewidth} p{0.32\linewidth} p{0.28\linewidth}}
\toprule
Пример & Доминирующая модель & Последствие \\
\midrule
Финансовые пузыри & Поведенческая & Нужна регуляция \\
Скандинавия (высокое доверие) & Социальная & Инвестиции в человеческий капитал \\
Инновации в бедных регионах & Ограниченная & Простые решения и обучение \\
\bottomrule
\end{tabular}

\vspace{4mm}
\begin{itemize}
  \item Разные модели → разные рекомендации по политике.
  \item Комбинация инструментов: стимулы + обучение + работа с нормами.
  \item Учитывать контекст и поведенческие особенности.
\end{itemize}
\note{Подытожить и дать практические рекомендации для политиков.}
\end{frame}

\begin{frame}{Спасибо!}
\centering
Вопросы?
\end{frame}

\end{document}