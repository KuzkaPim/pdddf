% Speaker notes (article) — развёрнутые заметки к презентации
% Компиляция: xelatex speaker_notes.tex
\documentclass[12pt]{article}
\usepackage{fontspec}
\setmainfont{TeX Gyre Termes}
\usepackage{geometry}
\geometry{a4paper, margin=2.5cm}
\usepackage{parskip}
\usepackage{hyperref}
\begin{document}

\begin{center}
  {\LARGE Пояснения к презентации: \\
  \textbf{Модели поведения человека и их роль в экономическом развитии}}\\[8pt]
  KuzkaPim
\end{center}

\section*{Слайд 1 — Титульный}
Кратко представьтесь, сформулируйте цель: показать влияние моделей поведения на экономическую политику и развитие.

\section*{Слайд 2 — Введение: основные модели}
- Рациональная: кратко — максимизация полезности.
- Ограниченная рациональность: эвристики, ограниченные вычисления.
- Поведенческая: эмоции, предвзятости, систематические отклонения.
- Социальная: нормы, доверие, сети.
Совет: подчеркнуть, что модели комбинируются в реальных ситуациях.

\section*{Слайд 3 — Роль моделей в экономике}
- Для каждой модели дать 1–2 практических примера (налоги, внедрение технологий, кредитные пузыри, нормы сбережений).
- Акцент: разные модели дают разные политические рекомендации.

\section*{Слайд 4 — Сравнительная таблица}
- Пройтись по строкам таблицы, показать где какая модель полезна.
- Пример: в задачах инфраструктуры рациональные стимулы важны; в вопросах сбережений — работаем с рамками и нормами.

\section*{Слайд 5 — Рациональная модель}
- Объяснить ключевые элементы: агент, информация, стимулы.
- Ограничения: несовершенная информация, транзакционные издержки.

\section*{Слайд 6 — Ограниченная рациональность и поведенческая экономика}
- Примеры эвристик: правило большого пальца, якорение.
- Влияние на пенсионные решения, потребление и кредит.

\section*{Слайд 7 — Социальные и культурные факторы}
- Доверие снижает издержки транзакций.
- Нормы могут ускорять или тормозить реформы.
- Пример: доверие в скандинавских странах → высокие инвестиции в человека.

\section*{Слайд 8 — Примеры и выводы}
- Подвести итоги: комбинировать инструменты, учитывать контекст.
- Предложить вопросы для обсуждения: какая модель доминирует в вашей стране?

\bigskip
Советы по подаче:
- Используйте локальные примеры.
- Вовлекайте аудиторию вопросами.
\end{document}